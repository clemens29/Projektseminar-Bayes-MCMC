% Latex document @author Clemens Näther, Jakub Kliemann, s85426, s85515
%--------------------------------------------------------------------
\documentclass[a4paper,12pt]{article}
\usepackage[utf8]{inputenc}
\usepackage{graphicx}
\usepackage[hidelinks]{hyperref}
\usepackage{float}
\usepackage{listings}
\usepackage{color}
\usepackage{amsmath}
\usepackage{pdfpages}
\usepackage{fancyhdr}
\usepackage{lastpage}
\usepackage{geometry}
\usepackage{listings}
\usepackage{xcolor}
\usepackage{mhchem}

\usepackage{biblatex} % Verwende BibLaTeX
\addbibresource{bibliography.bib} % Deine .bib-Datei

\lstset{ %
  language=,
  basicstyle=\ttfamily\small,
  backgroundcolor=\color{lightgray!20},
  frame=single,
  columns=fullflexible,
  breaklines=true,
  captionpos=b
}

\geometry{a4paper, top=25mm, left=30mm, right=25mm, bottom=30mm,
headsep=10mm, footskip=12mm}

\pagestyle{fancy}

\lhead{Dokumentation}

\rhead{Seite \thepage\ von \pageref{LastPage}}

\cfoot{}

\renewcommand{\headrulewidth}{0.4pt}

\renewcommand{\footrulewidth}{0.4pt}

\begin{document}

\begin{titlepage}

\begin{center}

\includegraphics[height=4cm]{../images/htwd-logo.jpg}\\[1cm]

\textsc{\LARGE Hochschule für Technik und Wirtschaft}\\[1.5cm]

\textsc{\Large Dokumentation}\\[0.5cm]

% Title
\newcommand{\HRule}{\rule{\linewidth}{0.5mm}}
\HRule \\[0.4cm]
{ \huge \bfseries \textsc{Projektseminar}}\\[0.4cm]
{ \huge \bfseries \textsc{Optimierung und Unsicherheitsquantifizierung mit Bayesianischer Statistik und MCMC-Methoden}}\\[0.4cm]
{ \huge \bfseries \textsc{(Prof. Schwarzenberger)}}\\[0.4cm]
\HRule \\[1.5cm]

% Author and supervisor
\begin{minipage}{0.4\textwidth}
\begin{flushleft} \large

\emph{Clemens Näther, s85426}\\
\emph{Jakub Kliemann, s85515}\\

\end{flushleft}
\end{minipage}
\end{center}
\end{titlepage}

\tableofcontents
\newpage

\section{Einleitung}
\newpage

\section{Theoretischer Teil}

\subsection{Grundlagen der bayesianischen Statistik und das Bayes'sche Theorem}
\subsubsection{Einführung in die bayesianische Statistik}
\subsubsection{Das Bayes'sche Theorem und seine Bestandteile}
%Überschrift%
\paragraph{Mathematische Formulierung}
%Text%

\begin{equation}
P(A|B) = \frac{P(B|A) \cdot P(A)}{P(B)}
\end{equation}

\paragraph{Prior- und Posterior-Verteilung}
%Text%

\paragraph{Likelihood-Funktion}
%Text%

\paragraph{Marginale Likelihood}
%Text%

\paragraph{Kombination von Vorwissen mit neuen Daten}
%Text%

\paragraph{Bezug zur Unsicherheitsquantifizierung}
%Text%


\subsubsection{Beispiele und praktische Anwendungen}

\newpage

\subsection{Binomiale Verteilung und deren bayesianische Interpretation}
\newpage

\subsection{Markov Chain Monte Carlo (MCMC) Methoden}
\newpage

\subsection{Konvergenzkriterien und Diagnosewerkzeuge für MCMC-Simulationen}
\newpage

\section{Praktischer Teil}

\subsection{Implementierung bayesianischer Modelle unter Verwendung in Python}
\newpage

\subsection{Anwendung der Modelle auf verschiedene Datensätze}
\newpage

\subsection{Durchführung von MCMC-Simulationen}
\newpage

\subsection{Interpretation der Ergebnisse}
\newpage

\subsection{Vergleich mit klassischen Methoden}
\newpage

\section{Zusammenfassung und Ausblick}
\newpage


\section{Literaturverzeichnis} 
Hier ist eine Zitation aus einem Buch \cite{BayesStatistik}. \\
Hier ist eine Zitation aus einem Buch \cite{EinfBayesStatistik}. \\
Hier ist eine Zitation aus einem Buch \cite{MonteCarloAlgorithmen}. \\
Hier ist eine Zitation aus einem Buch \cite{StatistikKlassischOderBayes}. \\
Hier ist eine Zitation aus einem Buch \cite{EinfBayesStatistikOptimalerStichprobenumfang}. \\

\printbibliography 
\newpage

\section{Selbstständigkeitserklärung}



\end{document}

